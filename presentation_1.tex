%%%%%%%%%%%%%%%%%%%%%%%%%%%%%%%%%%%%%%%%%
% Beamer Presentation
% LaTeX Template
% Version 1.0 (10/11/12)
%
% This template has been downloaded from:
% http://www.LaTeXTemplates.com
%
% License:
% CC BY-NC-SA 3.0 (http://creativecommons.org/licenses/by-nc-sa/3.0/)
%
%%%%%%%%%%%%%%%%%%%%%%%%%%%%%%%%%%%%%%%%%

%----------------------------------------------------------------------------------------
%	PACKAGES AND THEMES
%----------------------------------------------------------------------------------------

\documentclass{beamer}


\mode<presentation> {

% The Beamer class comes with a number of default slide themes
% which change the colors and layouts of slides. Below this is a list
% of all the themes, uncomment each in turn to see what they look like.

%\usetheme{default}
%\usetheme{AnnArbor}
%\usetheme{Antibes}
%\usetheme{Bergen}
%\usetheme{Berkeley}
%\usetheme{Berlin}
%\usetheme{Boadilla}
%\usetheme{CambridgeUS}
%\usetheme{Copenhagen}
%\usetheme{Darmstadt}
%\usetheme{Dresden}
%\usetheme{Frankfurt}
%\usetheme{Goettingen}
%\usetheme{Hannover}
%\usetheme{Ilmenau}
%\usetheme{JuanLesPins}
%\usetheme{Luebeck}
\usetheme{Madrid}
%\usetheme{Malmoe}
%\usetheme{Marburg}
%\usetheme{Montpellier}
%\usetheme{PaloAlto}
%\usetheme{Pittsburgh}
%\usetheme{Rochester}
%\usetheme{Singapore}
%\usetheme{Szeged}
%\usetheme{Warsaw}

% As well as themes, the Beamer class has a number of color themes
% for any slide theme. Uncomment each of these in turn to see how it
% changes the colors of your current slide theme.

%\usecolortheme{albatross}
%\usecolortheme{beaver}
%\usecolortheme{beetle}
%\usecolortheme{crane}
%\usecolortheme{dolphin}
%\usecolortheme{dove}
%\usecolortheme{fly}
%\usecolortheme{lily}
%\usecolortheme{orchid}
%\usecolortheme{rose}
%\usecolortheme{seagull}
%\usecolortheme{seahorse}
%\usecolortheme{whale}
%\usecolortheme{wolverine}

%\setbeamertemplate{footline} % To remove the footer line in all slides uncomment this line
%\setbeamertemplate{footline}[page number] % To replace the footer line in all slides with a simple slide count uncomment this line

%\setbeamertemplate{navigation symbols}{} % To remove the navigation symbols from the bottom of all slides uncomment this line
}

\usepackage{macros}

%----------------------------------------------------------------------------------------
%	TITLE PAGE
%----------------------------------------------------------------------------------------

\title[Tutorial on Bandit]{Tutorial on Bandit} % The short title appears at the bottom of every slide, the full title is only on the title page

\author{Subhojyoti Mukherjee} % Your name
\institute[IIT Madras] % Your institution as it will appear on the bottom of every slide, may be shorthand to save space
{
IIT Madras \\ % Your institution for the title page
\medskip
%\textit{john@smith.com} % Your email address
}
\date{\today} % Date, can be changed to a custom date

\begin{document}
\nocite{*}
\begin{frame}
\titlepage % Print the title page as the first slide
\end{frame}

\begin{frame}
\frametitle{Overview} % Table of contents slide, comment this block out to remove it
\tableofcontents % Throughout your presentation, if you choose to use \section{} and \subsection{} commands, these will automatically be printed on this slide as an overview of your presentation
\end{frame}

%----------------------------------------------------------------------------------------
%	PRESENTATION SLIDES
%----------------------------------------------------------------------------------------


\section{Introduction}
\input{intro}

\section{Stochastic Multi-Armed Bandit Problem}
\begin{frame}
\frametitle{Stochastic Multi-Armed Bandit Problem}
\begin{itemize}
\item<1-> In stochastic multi-armed bandit problem we are presented with a finite set of actions or arms. 
\item<2-> The rewards for each of the arms is drawn from identical and independent distributions. 
\item<3-> The learner does not know the mean of the distributions, denoted by $\mu_{i}$. 
\item<4-> The learner has to find the optimal arm the mean of whose distribution is denoted by $\mu^{*}$ such that $\mu^{*}> \mu_{i}, \forall i\in A$.
\item<5-> The distributions for each of the arms are fixed throughout the time horizon. 
\end{itemize}
\end{frame}

\begin{frame}
\frametitle{Basic Notations}
\begin{itemize}
\item<1-> Goal: To minimize Regret
\item<2->  Average reward of best action is $\mu^{*}$ and any other action $i$ as $\mu_{i}$. There are $K$ total actions. $T_{i}(n)$ is number of times tried action $i$ is executed till $n$-timesteps.
\item<3->  Cumulative Regret: The loss we suffer because of not pulling the optimal arm till the total number of timesteps  $n$. 
\begin{align*}
R_{n}=r^{*}n - \sum_{i\in A} r_{i}T_{i}(n),
\end{align*}
\item<4->  The expected regret of an algorithm after $n$ rounds can be written as
\begin{align*}
\E[R_{n}]= \sum_{i=1}^K \E[T_{i}(n)] \Delta_i,
\end{align*}
\item<5-> $\Delta_{i}=r^{*}-r_{i}$ denotes the gap between the means of the optimal arm and of the $i$-th arm. 
\end{itemize}
\end{frame}

\section{UCB1 Algorithm}
\begin{frame}
\frametitle{UCB 1 Algorithm}
\begin{algorithm}[H]
\caption{UCB1}
\begin{algorithmic}[1]
\State Pull each arm once
 \For{$t=K+1,..., n$}
\State Pull the arm such that $\max_{i\in A}\bigg\lbrace\bar{X}_{i} + \sqrt{\dfrac{2\log t}{s_i}}\bigg\rbrace$
 \EndFor
\end{algorithmic}
\end{algorithm}
\cite{auer2002finite}
\end{frame}

\section{Concentration Bounds}
\begin{frame}
\frametitle{Concentration Bounds}
\begin{itemize}
\item<1-> The issue of coin tossing.
\item<2-> Chernoff-Hoeffding Bounds and its applications.
\item<3-> Let $X_{1}, . . . , X_{n}$ be random variables with common
range [0, 1] and such that $E[X_{t} |X_{1}, . . . , X_{t-1}] = \mu.$ Let $\bar{S_n} = \dfrac{X_{1} +,....,+ X_{n}}{n}$. Then for all $a \geq 0$,
\begin{align*}
\mathbb{P} \lbrace \bar{S_{n}} \geq \mu + a \rbrace \leq e^{-2a^{2}n}\\
\mathbb{P}\lbrace \bar{S}_{n} \leq \mu - a \rbrace \leq e^{-2a^{2}n}
\end{align*}
\end{itemize}
\end{frame}


\section{UCB1 Theorem and Proof}
\begin{frame}
\frametitle{UCB1 Theorem on Regret Bound}
\begin{theorem}
For all $K > 1$, if policy UCB1 is run on K arms having arbitrary reward
distributions $P_{1}, . . . , P_{K}$ with support in $[0, 1]$, then its expected regret after any number $n$ plays is at most,
\begin{align*}
\mathbb{E}[R_n]\leq \sum_{i\in A}\dfrac{8\log n}{\Delta_{i}} + \sum_{i\in A}\Delta_{i}\bigg(1+\dfrac{\pi^{2}}{3}\bigg) 
\end{align*}
where $\mu_{1}, . . . , \mu_{K}$ are the expected values of $P_{1}, . . . , P_{K}$ .
\end{theorem}
\end{frame}

\begin{frame}
\frametitle{UCB1 Proof}
\begin{itemize}
\item<1-> The main goal is to bound the number of pulls ($T_i(n)$) of the sub-optimal arm $i$ till the $n$-th timestep.
\item<2-> So, we will assume that the $i$-th arm has been pulled atleast $\ell$ times and bound the probability of how many times it can be pulled after that.
\begin{align*}
T_{i}(n)&\leq \ell +\sum_{t=K+1}^{n}\lbrace I_t=i, T_i(t-1)\geq \ell\rbrace
\end{align*}
\item<3-> But, this is nothing but the probability that how many times after the $\ell$ pulls the UCB of $*$ is less than the UCB of $i$ which will have the highest UCB among all arms in $A$ to be selected,
\begin{align*}
T_{i}(n)&\leq \ell + \sum_{t=1}^{\infty}\sum_{s=1}^{t-1}\sum_{s_i =\ell}^{t-1}\lbrace \bar{X}_{s}^{*} + c_{t,s} \leq \bar{X}_{i,s_i} + c_{t,s_i} \rbrace
\end{align*}
\end{itemize}
\end{frame}

\begin{frame}
\frametitle{UCB1 Proof}
\begin{itemize}
\item<1-> The main argument lies in this, 
\includegraphics[scale=0.19]{img/UCB1_pic} 
\item<1-> \begin{align*}
\bar{X}^{*}_{s}&\leq \mu^* - c_{t,s}\\
\bar{X}_{i,s_i}&\geq \mu + c_{t,s_i}\\
\mu^* & < \mu_i + 2c_{t,s_i}
\end{align*}
\end{itemize}
\end{frame}

\begin{frame}
\frametitle{UCB1 Proof}
\begin{itemize}
\item<0-> Now, we get the value of confidence interval $c_{t,s_i}=\sqrt{\dfrac{2\ln t}{s_i}}$ by plugging its value in the below equations,
\item<1-> $\mathbb{P}\lbrace  \bar{X}^{*}_{s}\leq \mu^* - c_{t,s}\rbrace\leq \exp\bigg(-2\big(\sqrt{\dfrac{2\ln t}{s}}\big)^2 s\bigg) \leq e^{-4\log t} \leq t^{-4}$
\item<2-> $\mathbb{P}\lbrace \bar{X}_{i,s_i} \geq \mu + c_{t,s_i}\rbrace\leq \exp\bigg(-2\big(\sqrt{\dfrac{2\ln t}{s_i}}\big)^2 s_i\bigg) \leq e^{-4\log t} \leq t^{-4}$
\item<3-> And by plugging $\ell=\bigg\lceil \dfrac{8\log n}{\Delta_{i}^{2}}\bigg\rceil$ in,
\begin{align*}
\mu^* - \mu_i - 2c_{t,s_i} = \mu^* - \mu_i - 2\sqrt{\dfrac{2\log t}{s_i}} \geq \mu^* - \mu_i -\Delta_i =0
\end{align*} 
we get $\mu^* - \mu_i - 2c_{t,s_i} \geq 0$. So for any pulls greater than $\ell$, $\mu^*$ will surely be atleast $2c_{t,s_i}$  more than $\mu_i$ and one of the rest two events will occur with high probability.
\end{itemize}
\end{frame}

\begin{frame}
\frametitle{UCB1 Proof}
\begin{itemize}
\item<1-> Summing everything up, any sub-optimal arm $i$ will get pulled atleast $\ell$ times and then the two events $\bar{X}^{*}_{s}\leq \mu^* - c_{t,s}$ and $\bar{X}_{i,s_i}\geq \mu + c_{t,s_i}$ will occur with atmost $t^{-4}$ probability.
\item<2-> \begin{align*}
\mathbb{E}[T_{i}(n)]&\leq \bigg\lceil \dfrac{8\log n}{\Delta_{i}^{2}}\bigg\rceil + \sum_{t=1}^{\infty}\sum_{s=1}^{t-1}\sum_{s_i =\ell}^{t-1}2t^{-4}\\
&\leq \dfrac{8\log n}{\Delta_{i}^{2}} +1 + \dfrac{\pi^{2}}{3}, \text{by Bazel's equation}
\end{align*}
\item<3-> So finally the cumulative regret is,
\begin{align*}
\mathbb{E}[R_n]&\leq \sum_{i\in A}\mathbb{E}[T_i (n)]\Delta_i
\leq \dfrac{8\log n}{\Delta_{i}} + \Delta_{i}\bigg(1 + \dfrac{\pi^{2}}{3}\bigg)
\end{align*}
\end{itemize}
\end{frame}

\section{PAC Guarantees}
\begin{frame}
\frametitle{Looking beyond Cumulative regret}
\begin{itemize}
\item<1-> Sometimes rather than worrying about cumulative regret we just want to be satisfied with a \emph{nearly good} arm with \emph{high probability}.
\item<2-> This is called the $(\epsilon,\delta)$-guarantee or PAC-guarantee. 
\item<3-> We are interested in finding an arm such that it's $\epsilon$ close to the optimal arm and we can guarantee this with $1-\delta$ probability.
\item<4-> Note, that $\epsilon$ and $\delta$ are given as input and the main aim is to \emph{minimize the number of pulls of an arm $i$ so that it is $\epsilon$ close to the optimal arm with $1-\delta$ probability}. This is called Sample complexity. 
\end{itemize}
\end{frame}

\begin{frame}
\frametitle{A Naive Algorithm}
\begin{algorithm}[H]
\caption{Naive Algorithm}
\begin{algorithmic}[1]
\State Input: $\epsilon > 0$, $\delta > 0$
\State Output: An arm  
\For{each arm $i\in A$}
\State Sample it for $\ell=\dfrac{4}{\epsilon^{2}}\log \dfrac{2K}{\delta}$
\State Let $\bar{X}_i$ be the average reward of arm $i$
\EndFor
\State Output $argmax_{i\in A}\lbrace \bar{X}_i \rbrace$
\end{algorithmic}
\end{algorithm}
\cite{even2006action}
\end{frame}

\begin{frame}
\frametitle{Sample Complexity of Naive Algorithm}
\begin{theorem}
The sample complexity of Naive Algorithm for a set of arms $K$ is given by,
\begin{align*}
O\bigg( \dfrac{K}{\epsilon^2}\log \big( \dfrac{K}{\delta} \big) \bigg)
\end{align*}
\end{theorem}
\cite{even2006action}
\end{frame}

\begin{frame}
\frametitle{Naive Algorithm Sample Complexity Proof}
\begin{itemize}
\item<1-> We want to bound the probability of the event $\bar{X}_i > \bar{X}^* $. The goal is to find the minimum number of pulls required for an arm $i$ so that $\mu^{*}-\mu_i < \epsilon$.
%But by $(\epsilon,\delta)$ definition we need $i$ only to be $\epsilon$ close to $*$.
%Let $\epsilon$ be such that $\mu^* - \mu_i < \epsilon$.
\item<2-> So, we need to bound the opposite condition for sample complexity because till that time we need to pull $i$. Let $i$ be an arm such that $\mu_i<\mu^* - \epsilon \rightarrow \mu_i + \dfrac{\epsilon}{2} <\mu^* - \dfrac{\epsilon}{2}$.
\item<3-> So, we only need to bound the probability of, 
\begin{align*}
\mathbb{P}\lbrace\bar{X}_i > \bar{X}^* \rbrace &\leq \mathbb{P}\bigg\lbrace \bar{X}_i > \mu_i + \dfrac{\epsilon}{2}\bigg\rbrace + \mathbb{P}\bigg\lbrace \bar{X}^* < \mu^* - \dfrac{\epsilon}{2}\bigg\rbrace\\
&\leq 2\exp\bigg(-2\big (\dfrac{\epsilon}{2}\big)^{2}\ell\bigg)\leq 2\exp\bigg(-2\dfrac{\epsilon^2}{4}. \dfrac{4}{\epsilon^{2}}\log \dfrac{2K}{\delta} \bigg)\leq \dfrac{\delta}{K}
\end{align*}
\item<4-> Summing over all the $K-1$ arms (arms excluding $*$) we get, $\dfrac{(K-1)\delta}{K}< \delta$
\end{itemize}
\end{frame}

\begin{frame}
\frametitle{Intuition about Median Elimination}
\begin{itemize}
\item<1-> Can we be more powerful than using Naive Algorithm?
\item<2-> One simple way to modify Naive Algorithm is to divide the time horizon into phases.
\item<3-> In each phase pull all the surviving arms equal number of times.
\item<4-> After that eliminate half the surviving arms with a high guarantee that they are surely not $\epsilon$-optimal arms.
\end{itemize}
\end{frame}

\begin{frame}
\frametitle{Median Elimination}
\begin{algorithm}[H]
\caption{Median Elimination}
\begin{algorithmic}[1]
\State Input: $\epsilon > 0$, $\delta > 0$
\State Output: An arm  
\State Set $S_1 =A$, $\epsilon_1 = \epsilon/4$, $\delta_1=\delta/2$ and $\ell=1$
\For{Repeat till $|S_{\ell}|=1$}
\State Sample every arm in $S_\ell$ for $\dfrac{4}{\epsilon_{\ell}^{2}}\log(\dfrac{3}{\delta_\ell})$ times and let $\bar{X}_i$ denote the average estimated payoff of $i$.
\State Find median $m_{\ell}$ of all surviving arms based on their $\bar{X}_{i},\forall i\in S_\ell$
\State Eliminate all arms from $S_{\ell}$ such that $\bar{X}_{i}< m_{\ell}$ and create $S_{\ell+1}$.
\State Reset Parameters: $\epsilon_{\ell+1}=\dfrac{3}{4}\epsilon$; $\delta_{\ell+1}=\dfrac{1}{2}\delta$; $\ell=\ell+1$
\EndFor
\end{algorithmic}
\end{algorithm}
\cite{even2006action}
\end{frame}

\begin{frame}
\frametitle{Comparison of Median Elimination, Naive Algorithm}

\begin{table}
\caption{Sample Complexity of Median Elimination, Naive Algorithm}
\begin{center}
\begin{tabular}{|c|c|}
\toprule
Algorithm  & Upper bound on Sample Complexity \\
\midrule
Naive        &$O\bigg(\dfrac{K}{\epsilon^2}\log \big( \dfrac{K}{\delta} \big) \bigg)$ \\\midrule
ME      &$O\bigg(\dfrac{K}{\epsilon^2}\log \big( \dfrac{1}{\delta} \big)  \bigg)$\\\bottomrule
\end{tabular}
\end{center}
\end{table}

\begin{itemize}
\item<1-> So clearly Naive algorithm uses more samples than Median Elimination to give us the same $(\epsilon,\delta)$ guarantee
\end{itemize}

\end{frame}



\section{Arm Elimination Algorithm}
\begin{frame}
\frametitle{Arm Elimination Algorithm}
\begin{itemize}
\item<1-> Arm Elimination (AE) algorithms proceeds in a phase-wise manner whereby they pull all the surviving arms equal number of times in each phase and then at the end of the phase proceeds to eliminate one or more arms from its active set. 
\item<2-> We have seen that Median Elimination algorithm which is an AE algorithm, is more powerful than Naive Algorithm.
\item<3-> Can we have such algorithm for minimizing Cumulative Regret?
\end{itemize}
\end{frame}

\begin{frame}
\frametitle{Enter UCB-Improved}
\begin{itemize}
\item<1-> The basic idea of UCB-Improved (\cite{auer2010ucb}) is nearly same as ME.
\item<2-> Divide the horizon into phases and initialize parameters.
\item<3-> Pull all surviving arms equal number of times during a phase.
\item<4-> At the end of the phase eliminate some arms based on some criteria.
\item<5-> Reset parameters and proceed to next phase.
\end{itemize}
\end{frame}

\begin{frame}
\frametitle{Enter UCB-Improved}
\begin{algorithm}[H]
\caption{UCB-Improved}
\begin{algorithmic}[1]
\State {\bf Input:} Time horizon $n$
\State {\bf Initialization:} Set $B_{0}:=A$ and $\epsilon_{0}:=1$.
\For{$m=0,1,..\big \lfloor \dfrac{1}{2}\log_{2} \dfrac{n}{e}\big\rfloor$}	
\State Pull each arm in $B_m$ so that the total number of times it has been pulled is $n_{m}=\bigg\lceil\dfrac{2\log{( n\epsilon_{m}^{2})}}{\epsilon_{m}}\bigg\rceil$. 
\ArmElim
\State For each $i \in B_{m}$, delete arm ${i}$ from $B_{m}$ if,
\begin{align*}
\bar{X}_{i} + \sqrt{\dfrac{\log{(n\epsilon_{m}^{2})}}{2 n_{m}}}  < \max_{{j}\in B_{m}}\bigg\lbrace\bar{X}_{j} -\sqrt{\dfrac{\log{( n\epsilon_{m}^{2})}}{2 n_{m}}} \bigg\rbrace
\end{align*}
\EndArmElim
\ResParam
\State Set $\epsilon_{m+1}:=\dfrac{\epsilon_{m}}{2}$, Set $B_{m+1}:=B_{m}$
\EndResParam
\State Stop if $|B_{m}|=1$ and pull ${i}\in B_{m}$ till $n$ is reached.
\EndFor
\end{algorithmic}
\end{algorithm}
\end{frame}

\begin{frame}
\frametitle{Comparison of UCB-Improved,ME and UCB1}
\begin{table}
\caption{Cumulative Regret of UCB-Improved, UCB1}
\begin{center}
\begin{tabular}{|c|c|}
\toprule
Algorithm  & Upper bound on Cumulative Regret\\
\midrule
UCB1        &$O\bigg(\dfrac{K\log n}{\Delta} \bigg)$ \\\midrule
UCB-Improved      &$O\bigg(\dfrac{K\log (n \Delta^2)}{\Delta} \bigg)$\\\bottomrule
\end{tabular}
\end{center}
\end{table}

\begin{itemize}
\item<1-> So, UCB-Improved is more powerful than UCB1 theoretically.
\item<2-> UCB-Improved is more powerful than ME, because ME will always take $\log_{2} K$ number of phases to come up with the best arm (since it eliminates half the number of arms after every phase) whereas UCB-Improved eliminates arbitrary number of arms in any phase.
\item<3-> Empirically, UCB-Improved beats UCB1 when $K$ is very large and gaps ($\Delta_{i},\forall i\in A$) are very small.
\end{itemize}
\end{frame}

\begin{frame}
\frametitle{Finally, an experiment!!!}
\begin{figure}
    \centering
    \begin{tabular}{c}
    \subfigure[0.32\textwidth][Experiment $1$: $100$ Gaussian-distributed arms with $\mu_{i_{{i}\neq {*}:1-33}}=0.01$, $\mu_{i_{{i}\neq {*}:34-99}}=0.06$, $r^{*}_{i=100}=0.1$ and $\sigma_{i:1-100}^{2} = 0.3$]
    {
    		\pgfplotsset{
		tick label style={font=\Large},
		label style={font=\Large},
		legend style={font=\Large},
		}
        \begin{tikzpicture}[scale=0.7]
      	\begin{axis}[
		xlabel={timestep},
		ylabel={Cumulative Regret},
		grid=major,
        %clip mode=individual,grid,grid style={gray!30},
        clip=true,
        %clip mode=individual,grid,grid style={gray!30},
  		legend style={at={(0.5,-0.2)},anchor=north, legend columns=3} ]
      	% UCB
		\addplot table{results/Expt1/UCB1comp_subsampled.txt};
		\addplot table{results/Expt1/UCB_Improvedcomp_subsampled.txt};
		\addplot table{results/Expt1/Med_Elimcomp_subsampled.txt};
      	\legend{UCB1,UCB-Improved,Med-Elim}
      	\end{axis}
      	\end{tikzpicture}
  		%\label{Fig:budgetExpt1}
    }
    \end{tabular}
    \caption{Experiment with bandit}
    %\label{fig:budgetExpt}
\end{figure}
\end{frame}


\section{Some Other Bandits} 
\begin{frame}
\frametitle{Some Other Bandits and Applications}

\begin{itemize}
\item<1-> Adversarial Bandits : In adversarial bandits we consider that the underlying distribution is non-stationary. Used in Investment in Stock Markets
\item<2-> Contextual Bandits : In this setup we consider that there is a context, a feature vector associated with each arm that is revealed once the arm is pulled. Based on that the learner must predict the next arm to pull. Used in online Advertisement/news article selection
\item<3-> Pure exploration Bandits : In this setup the sole consideration is to conduct as much exploration as possible within a limited number of pulls and then suggest the best arm(s). Used in Clinical trials.
%\item Distributed Bandits : Used in packet routing through a network
\end{itemize}
\end{frame}

\section{References}
\begin{frame}[allowframebreaks]
\frametitle{References}
\bibliographystyle{plainnat} 
\bibliography{biblio}
\end{frame}


%------------------------------------------------

\begin{frame}
\Huge{\centerline{Thank You}}
\end{frame}

%----------------------------------------------------------------------------------------

\end{document} 